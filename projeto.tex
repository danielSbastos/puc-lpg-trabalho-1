%%%%%%%%%%%%%%%%%%%%%%%%%%%%%%%%%%%%%%%%%%%%%%%%%%%%%%%%%%%%%%%%%%%%%%
% How to use writeLaTeX: 
%
% You edit the source code here on the left, and the preview on the
% right shows you the result within a few seconds.
%
% Bookmark this page and share the URL with your co-authors. They can
% edit at the same time!
%
% You can upload figures, bibliographies, custom classes and
% styles using the files menu.
%
%%%%%%%%%%%%%%%%%%%%%%%%%%%%%%%%%%%%%%%%%%%%%%%%%%%%%%%%%%%%%%%%%%%%%%

\documentclass[12pt]{article}

\usepackage{sbc-template}
\usepackage{listings}
\usepackage{hyperref}
\usepackage{graphicx,url}
\usepackage{fancyvrb}
\usepackage[pdftex]{hyperref}
\usepackage{subfig}
\usepackage[utf8]{inputenc}  

\hypersetup{
    colorlinks=true,
    urlcolor=blue,
}
    
\sloppy

\title{Trabalho sobre Sintaxe e Semântica de Linguagens de Programação}
\author{Daniel S. Bastos\inst{1}, Lucas M. S. Freitas\inst{1}, Rossana O. Souza \inst{1} }
\address{Pontificia Universidade Católica de Minas Gerais
(PUC MG)\\
Rua Cláudio Manuel, 1.162 -- Funcionarios -- Belo Horizonte -- MG -- Brazil
}

\begin{document} 

\maketitle

\section{Introdução}

Esse texto é um guia introdutório da linguagem R, com o intuito de brevemente explicar os conteúdos de aplicações, estrutura de dados, tipos, paradigmas, estruturas de controle, funções e tópicos avançados, como concorrência e paralelismo, presentes na linguagem.

O repositório da aplicação prática está localizado no \href{https://github.com/danielSbastos/puc-lpg-trabalho-1}{Github}, no qual o arquivo contendo o código é o \texttt{main.R}, e o vídeo final no \href{https://youtu.be/iBmmmesPYok}{YouTube}.

\section{Aplicações da Linguagem R} \label{sec:Aplicações da Linguagem R}
Essencialmente a linguagem R foi desenvolvida para a estatística, dessa forma ela possui inúmeras estruturas que auxiliam no desenvolvimento, tais como técnicas estatísticas de modelagem, classificação, testes estatísticos, agrupamento e técnicas gráficas para construção de gráficos.

Possui também uma gama de símbolos matemáticos e a possibilidade de representar números imaginários. Estes aspectos fazem a linguagem se destacar em áreas além da estatísica, como a de análise de dados e machine learning.

R é uma linguagem funcional com suporte para orientação a objetos (OO), podendo assim ser considerada multiparadigma. Tudo em R é um objeto, então sempre há uma constante interação entre o paradigma funcional e OO.

\section{Caraterísticas da Linguagem} \label{sec:Caraterísticas da Linguagem}
A linguagem R possui 6 tipos de dados primitivos ou básicos, sendo eles: character, numeric, integer, local e complex.

A declaração e definição de uma variável são realizadas simultâneamente, não é possível declarar uma variável sem definí-la ao mesmo tempo.

Maneira incorreta:
\begin{verbatim}
    a
    a <- 2
\end{verbatim}

Maneira correta:
\begin{verbatim}
    a <- 2
    a
\end{verbatim}

\subsection{Type Binding}
    R, assim como JavaScript, utiliza associação estática de tipo implicíta, ou seja, não é necessário uma instrução definindo o tipo da variável, as variáveis são associadas ao tipo atribuído à elas no momento em que são declaradas.

\begin{verbatim}
    a <- "123" # character
    b <- 123   # numeric
    c <- 123.5 # numeric (double)
    d <- 123L  # integer
\end{verbatim}
    Agora, vejamos um exemplo mais curioso:
\begin{verbatim}
    a = 10
    a = "Linguagens de Programação"
\end{verbatim}

Nesse caso a variável \texttt{a} sofreu uma modificação, não apenas de valor, mas de tipo, valor e tamanho. Ela passou de inteiro para string sem maiores problemas. Por ser uma linguagem não tipada a reatribuição de variáveis é algo muito comum e muito simples de ser implementado.

\subsection{Estruturas de Dados}
R possui 5 estruturas de dados, sendo elas: Vector, List, Matrix, DataFrame e Factors.

\subsection{Vector}
Estrutura de dados mais simples, armazena um conjunto de dados que pode ser de qualquer tipo, numeric, integer, character, logical ou complex. Um vetor pode armazenar por vez apenas um tipo de dado.
    Para declarar um vetor é necessário utilizar a função \texttt{c()}:
    
    \begin{verbatim}
    a <- c(1, 2, 3, 4)   # Vetor do tipo numeric
    b <- c("1", 2, 3, 4) # Tudo será convertido para char
    c <- c("f", 2, 3, 4) # Isso irá provocar um erro
    \end{verbatim}

Como mencionado anteriormente, tudo em R é um objeto e como esta é uma linguagem do paradigma OO é possível utilizar-se de um vetor para criar uma classe com atributos. Por exemplo:

    \begin{verbatim}
    d <- c(nome = "Maria", idade = "19") 
    class(d) <- "aluna"
    attributes(d)
    #   $names
    #   [1] "nome"  "idade"
    #   $class
    #   [1] "aluna"
    \end{verbatim}

\subsection{List}
Listas são estruturas mais complexas que vetores, por armazenar diferentes tipos de dados de uma única vez, até mesmo vetores dentro delas.
    Para declarar uma lista é necessário utilizar a função \texttt{list()}:
\begin{verbatim}
b <- list(12, "Samuel", 3, 6)
str(b)
    # List of 4
    # $ : num 12
    # $ : char "Samuel"
    # $ : num 3
    # $ : num 6
    
b <- list(c(11, 23, 53, 40), "Sara", 4) # lista com vetor
b[[1]][1]
    # 11
\end{verbatim}

\subsection{Matrix}
    Matrizes são bem parecidas com vetores, assim como eles uma matriz só pode conter um tipo de dado por vez, sua maior diferença está na dimensão, enquanto vetores são de uma dimensão, matrizes podem ser de uma ou mais dimensões, dessa forma contendo linhas e colunas.
    Utiza-se da função \texttt{matrix(valores, linhas)} para declarar uma matriz: \texttt{m <- matrix(1:9, nrow=3)}
    
\subsection{DataFrame}
    DataFrame é uma das estruturas mais complexas da linguagem. De maneira sucinta, funciona como uma matriz que aceita varios tipos de valores. Esse tipo de estrutura é muito utilizado para ler grandes volumes de dados. Para declarar um DataFrame é necessário ler uma base de dados através da função \texttt{read}: \texttt{df <- read.csv("arquivo.csv")}

\subsection{Factor}
 Fator é um tipo que categoriza e conta a ocorrência de valores, por exemplo:
 \begin{verbatim}
vetor <- c(400, 550, 300, 400) # isso é um vetor
fator <- as.factor(vetor) # converter para o tipo factor
summary(fator)
    # 300 400 550 -> valores categorizados
    # 1   2   1   -> quantidade de vezes que cada um apareceu
 \end{verbatim}

\section{Escopo Léxico} \label{sec:Escopo Lexico}
    R utiliza o escopo léxico (ou estático), portanto, os valores de variáveis são definidos pelo seu ambiente de nível superior, e não pela pilha de chamadas, exemplo:
    \begin{verbatim}
y <- 2
f <- function(x) {
    y**2 + x
}
g <- function(x) {
    y <- 10
    f(x)
}
g(3)
# 7
    \end{verbatim}
Sendo o valor de x = 3 o resultado será: 7. Caso o escopo fosse dinâmico, o resultado seria 103.

\section{Estruturas de Controle} \label{sec:Comandos de Execução}

Há dezenas de comandos de execução e estruturas de controle em R, portanto, esse texto atêm-se aos mais populares. 
\\
\begin{table}[h]
\centering
\caption{Operadores lógicos e relacionais}
 \subfloat[]{
 \centering
 \begin{tabular}{|c|c|c|}
 \toprule
 Relação&Operador&Emprego\\ \midrule
 \hline
 Igual & \begin{math} == \end{math} & \begin{math}5 == 3\end{math}\\
 Diferente &\begin{math} != \end{math} & \begin{math}5 != 3\end{math}\\
 Menor & \begin{math} < \end{math} & \begin{math}5 < 3\end{math}\\
 Menor e igual &\begin{math} <= \end{math} & \begin{math}5 <= 3\end{math}\\
 Maior & \begin{math} > \end{math} & \begin{math}5 > 3\end{math} \\
 Maior e igual & \begin{math} >= \end{math} & \begin{math}5 >= 3\end{math}\\ \bottomrule
 \end{tabular}}
 \hspace{0cm}
 \subfloat[]{
 \centering
 \begin{tabular}{|c|c|c|}
 \toprule
 Relação&Operador&Emprego\\ \midrule
 \hline
 E & \begin{math} \&\& \end{math} & \begin{math}X>10 \&\& Y>0 \end{math}0\\
 OU &\begin{math} || \end{math} & \begin{math} X>10  || Y<0 \end{math}\\
 NÃO & \begin{math} ! \end{math} & \begin{math} !(X>10) \end{math}\\ \bottomrule
 \end{tabular}}
\end{table}

Estrutura Condicional (if) e Laço Iterador (for e while):

\begin{verbatim}
    if(condição){
        <instrução se TRUE>
    }
    for (<variável> in <set>){
        <instruções>
    }
    while(<condição>){
        <instrução enquanto TRUE>
    }
\end{verbatim}

Uma estrutura de repetição não comum com outras linguagens mais populares é o comando repeat. Nesse caso o loop também é executado enquanto as condições forem \texttt{true}. É possível interromper a repetição utilizando-se \texttt{break} ou \texttt{next}.

\begin{verbatim}
    repeat {
    <instrução antes da condição 1>
    if(<condição 1>) break
        <instrução antes da condição 2>
    if(<condição 2>) break
        <instrução depois da condição 2>
    }
\end{verbatim}

Veja agora uma estrutura que é comum com a linguagem tipada C. A estrutura de casos switch pode representar \texttt{if else if} planas. A condição de cada caso é sempre uma igualdade e geralmente o switch é usado para igualdade de valores entre strings.

\begin{verbatim}
    switch(<variável>,
        <valor 1> = { <instrução da condição 1> },
        <valor 2> = { <instrução da condição 2> },
        { <instrução da última condição> }
\end{verbatim}

\section{Funções} \label{sec:Funções}

Funções são uma construção muito poderoso da linguagem R, elas atuam como objetos e, portanto, podem ser usadas como parâmetro e são tratadas como "objetos de primeira classe". É possível ainda usar funções dentro de funções.

Um importante fator que caracteriza uma linguagem funcional é ter estes objetos de primeira classe e de também alta ordem. No exemplo abaixo, para cada item da lista é calculado a média, também há um objeto de primeira classe, a função \texttt{mean}, e o de alta ordem, a função \texttt{lapply}.

\begin{verbatim}
    lapply(list(1:10, 1:4, 1:30), mean)
    # [[1]]
    # [1] 5.5
    # [[2]]
    # [1] 2.5
    # [[3]]
    # [1] 15.5
\end{verbatim}

Algumas operações são tão recorrentes que já possuem funções prontas para executá-las. Há muitos pacotes e bibliotecas do R que facilitam muito as tarefas do programador com códigos de ações frequentes prontos. Veja o exemplo:

\begin{verbatim}
    a = "João"
    b = "Silva"
    c = c(a, " ", b) # c = João Silva
\end{verbatim}

No exemplo acima a ação de combinar valores pode ser feita através do pacote "stringr" com a utilização da função \texttt{c}. Com ela é possível combinar indefinidos valores passando-os como parâmetros, sejam eles variáveis ou textos, como foi feito com o espaço entre o nome e o sobrenome no exemplo mostrado anteriomente. No exemplo a seguir pode-se observar um outro tipo função, a função recursiva.

Função não recursiva

\begin{verbatim}
fatorial = function(n) {
  if (n%%1!=0 || n<0) stop("erro")   
  else if (n == 0) return(1)
  aux=1
  for (i in 1:n) {
    aux=i*aux
  }
  return(aux)
}
\end{verbatim}

Função  recursiva

\begin{verbatim}
fatorialrec = function(n){
  if (n%%1!=0 || n<0) stop("erro")
  else if (n==1 || n==0) return(1)
  return(fatorialrec(n-1) * n)
}
\end{verbatim}

Como pode ser observado, a recursividade se trata da capacidade da função de fazer uma chamada para si mesma, isso pode tornar o programa mais conciso, mas não necessariamente mais complexo. Perceba que, enquanto a função não recursiva utilizou sete linhas, a segunda se utilizou de apenas três. 

\section{Tópicos Avançados} \label{sec:Tópicos Avançados}
\subsection{Concorrência e Paralelismo}

Até agora todos os exemplos de código executam de forma sequencial. Para cenários onde a quantidade de dados é pequena e o tempo de execução não é um limitante, o processamento sequencial é suficiente, porém a medida que a quantidade e o tempo aumentam a ponto de se tornarem inviáveis, outras formas além do processamento sequencial devem ser exploradas, e essas são o paralelo e concorrente.

O pacote "future" provê ferramentas para processamento concorrente e paralelo. Este tem o conceito de representar uma operação que não é necessária no instante que foi definida, podendo assim ser executada em background quando requerida. O próprio pacote resolve e otimiza em qual momento uma tarefa deve pausar para outra executar, e quando essa pode resumir, assim dando suporte à concorrência. 

O exemplo abaixo utiliza o pacote "future"

\begin{verbatim}
    library(future)
    plan(multicore)
    fa = future({ 1 + 1 })      # 1
    fb = future({ 2 + 2 })      # 2
    v = c(value(fa), value(fb)) # 3
\end{verbatim}

Na linha 1 e 2 há duas operações , \texttt{fa} e \texttt{fb}, ambas não necessárias no momento que são definidas. Elas serão somente necessárias na linha 3, que irá executar de forma paralela em CPUs com mais de um núcleo. 

\section{References}
R PROGRAMMING for Data Science. [S. l.], 3 set. 2020. Disponível em: \url{https://bookdown.org/rdpeng/rprogdatascience}. Acesso em: 6 mar. 2021

STATISTICS 506, Fall 2016: Concurrent programming in R. [S. l.], 2016. Disponível em: \url{http://dept.stat.lsa.umich.edu/~kshedden/Courses/Stat506/r_concurency/}. Acesso em: 6 mar. 2021.

A FUTURE for R: A Comprehensive Overview. [S. l.], 2015. Disponível em: \url{https://cran.r-project.org/web/packages/future/vignettes/future-1-overview.html}. Acesso em: 6 mar. 2021.

ESTATÍSTICA é com R!. In: Estatística é com R!. [S. l.], 17 maio 2015. Disponível em: \url{http://www.estatisticacomr.uff.br/?p=98#:~:text=Recurs%C3%A3o%20ou%20recursividade%20pode%20ser,que%20seu%20similar%20sem%20recurs%C3%A3o}. Acesso em: 7 mar. 2021.

\end{document}
